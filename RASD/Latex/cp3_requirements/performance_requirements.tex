The scope of this section is to specify both the static and the dynamic numerical requirements placed on the software or on human
interaction with the software as a whole.
\paragraph{Static numerical requirements:}\mbox{}\\
By account of the European Automotive Manufacturer's Association (ACEA) in 2021 the car fleet of the whole EU amounted to 242 million of vehicles and among those 1.1\% are electrically-chargeable vehicles, in other words in EU roads there are roughly 2.7 million vehicles with a plug. The number of new registered vehicles amounted to 9.7 million, of which 18\%, roughly 1.7 million, (it was 10\% in 2020) were EVs or plug-in hybrids. This shows that the adoption of electrically-chargeable vehicles is increasing and so the need to use systems to manage the charging process. With these numbers in mind we can be cautious and consider that roughly a number of 15 million electrically-chargeable vehicles will be circulating in the EU roads in the next year, so we will base our analysis on this number. Obviously the results can be factored by the market share the company believes it will cover.
\par
We can assume that to each EV corresponds a user and that 1KB of data each is sufficient to record the personal data of the user and the information regarding the parameters of the EVD:
\[ userData = 15 * 10^6 * 1KB = 15 GB \]
\[ evData = 15 * 10^6 * 1KB = 15 GB \]
There is also the information regarding the charging stations and all their parameters, like charging points, prices, location, rating, reviews etc. In consideration of the low number of charging stations, that we assume to be no more than 100,000, and the generous amount of data needed to describe them we can allocate 1MB for charging stations which gives us:
\[chargingStationsData = 10*5 * 1MB = 100 GB\]
Furthermore we have to keep track of the charging history, in particular each element of the list must contain:
\begin{itemize}
    \item EV identifier
    \item charging station identifier
    \item timestamp of charging start
    \item timestamp of charging end
    \item price per kWh paid
    \item kWh recharged
\end{itemize}
1KB is large enough to encode this information. Limiting the length of the charging history to 100 elements we get a maximum storage need of:
\[ maxHistoryDataPerUser = 100 * 1KB = 100 KB\]
\[ maxHistory = 15 * 10^6 * maxHistoryDataPerUser = 1.5 TB\]
Which is perfectly manageable by modern system. But we also point out that this information can also be stored not in a central server, but in a distributed manner in the mobile devices of the users.
\par
\medskip
For the CPMS we have to keep track of all the information generated by the charging points for each charging station that it manages. We can assume that 1KB of data is enough to store all the information of charger and that a charging station has on average 10 chargers, so:
\[ chargersDataAvg = 10 * 1KB = 10 KB \]
We also have to keep track of the CPO authentication details but the memory footprint of this data is negligible so it will be not considered.
\par
Next we have to keep track also of the data that we get from the DSOs. The electricity grid market is characterized by high cost of this investment, which means that there are only a few organization that operate in this market, and we assume 100 being generous. Assuming also 1MB of data for each DSO we get:
\[ dsoData = 100 * 1MB = 100 MB \]
\medskip

\paragraph{Dynamical performance requirements:}\mbox{}\\
Considering that the operation of charging an EV is not an operation that happens that often, we assume that at a certain point of time only 2-5\% of the user are actively using the eMSP at the same moment. Since the speed of processing a request is not vital to the functionality of the system, we do not impose any hard constraint on the time a request should be processed, only that it shall be done on the order of 5-20 seconds to give the user the impression of good usage experience.
\medskip
\par
Regarding eMma, we'd like to point out that it's the component that will influence the most the perceived usage experience so, for all the operations that do not involve request to the eMSP server, eMma shall process the request in less than 2 seconds.
\medskip
\par
eMci, like eMma, shall process the transaction in less than 2 seconds for all those operations that do not require communication with the rest of the eMSP.
\medskip
\par
The CPMS dynamical performance requirements regarding the changing of the cost of charging, the usage of energy stored in batteries, and concluding purchase agreements with a new DSO may be fundamental for the economy of the business, so the operations regarding this aspect of managing the charging station shall be processed as soon as possible, in sub-second time after the CPO confirms the transaction. Furthermore the time required by the CPMS to unlock and start the charging process from a charger shall not be greater than 10S. 