Here we will specify the main communication interfaces that can be used to solve the communication and integration of different components.

\begin{enumerate}
    \item eMma and eMci are part of the eMSP that that run on separate machine from it. This means that they'll need to communicate using some kind of network protocol. In particular we shall use the internet stack of protocol. Surely up to the transport layer (TCP/UDP) the standard protocols shall be used, meanwhile the application layer shall be let to the architectural designer to decide weather to use a proprietary protocol or a standard one like \verb|http|.
    \item The eMSP needs to communicate with different CPMS. We can't make any assumption on how each different CPMS will provide their services so this part of the eMSP that handles the communication shall be constructed in the most general way. If, during further analysis, this implementation becomes too troublesome and complex than an assumption can be made and build the system such that it can communicate with only the CPMSs that use the \verb|OCPI| protocol.
    \item The CPMS needs to communicate with the software managing the charging points. An open standard protocol has been devised for this purpose and is widely adopted, the OCPP. Our system will be using this protocol to communicate with the charging points.
    \item The CPMS needs to communicate with different DSOs to get information about the price, source and other characteristics of the supplied electricity and to conclude a purchase agreement. For this kind of communication there is no open standard protocol, so an ad-hoc interface needs to be build to be able to interface with different DSOs.
\end{enumerate}