\subsection{Reliability}
It goes without saying that the system shall achieve an adequate level of reliability. In particular some functionalities of the system shall present stronger reliability than others. In particular the key functionalities of the system like login, booking, charging and payment must exhibit a fairly reliable behaviour. The software should be thoroughly tested before delivery to remove defect that would cause the system to crash. The system shall be implemented as loosely coupled as possible and the component that implement the key functionalities described above should be independent from the other part of the system. In this manner failures of secondary components shall not affect the correct working of the main ones. From the hardware point of view, to increase reliability, redundant and parallel architecture should be implemented when possible. In this way, a fail-over mechanism can be implemented if one of the machines fails.

\subsection{Availability}
Even though periods of downtime can cause disruption to clients, these are not critical, so interval of time where the service is not available may be tolerated. Still, a not working service may damage the image of the company and the business itself, thus availability is an aspect that should not be disregarded. The system shall, then, be build with replication and fault tolerance in mind. The initial availability target that the designer  should focus on achieving is two nines, meaning that the system should be downtime roughly for no more than 4 days in a year. Given that the devices in which the system will be running are fairly common and inexpensive, the system shall be build with extensions in mind by means of replication and fault tolerance. Following this approach, the availability score can be adjusted based on customer needs.

\section{Security}
eMall stores a lot of user sensitive data and these must be protected in the event of malicious attacks from external agents. To guarantee the protection of the data provided by the users, common security practices shall followed. In particular, sensitive data shall not be stored in plain-text but a it shall be encrypted using a secure encryption algorithm like SHA2. Furthermore, a lot of communication occurs through the internet  by different components of our system. We can't exclude the possibility of potential sniffing attacks on the communication network, so all communications shall be conducted with adequately secure communication protocols like \verb|https| or \verb|TLS|.

\subsection{Maintainability}
The charging facilities and management is a relative new and dynamically changing environment. This means that the system shall be built to facilitate maintenance and extensions. To achieve this goal the designer and the development team should adhere to the commonly known patterns and principles that guide to a modular, lowly coupled and high coherent system. It goes without saying that, the whole system needs to be thoroughly documented, both by comments on the code and by providing a specification document. % i heard someone calling them spec document

\subsection{Portability}
As we previously discussed, we mentioned that the eMma component of the system shall run on mobile devices. Currently there are two main platforms that dominate the market in this sector, but new competitors may arise in the future, so the option building this part of the system with tools that will guarantee it's portability to new platforms must be considered if cost-effective. While for the development of the eMci component, using tools that guarantee the portability will be mandatory due to high variety of embedded platforms present in the market.\\
The last note is about the eMSP server part and CPMS modules. Since our system hasn't any distribution requirement, portability is not really needed for these sub-systems.