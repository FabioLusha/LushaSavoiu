The eMall is modeled as a software with two possible user interfaces, one for the mobile application, which will be available to the users, and one for the web application available to the businesses, that offer the charging service.

The user interface of the eMall, that the EVDs interact with, is thought as a mobile application, the eMma, easy to use and intuitive, allowing users to quickly and easily access the features they need to charge their vehicles. The EVD needs to download the mobile app on his cellphone in order to interact with the eMall and take advantage of its functionalities.
We want the application to be, also, visually appealing and easy to navigate, with well-designed buttons, menus, and other elements that make it easy for users to find the information they need and interact inserting the necessary data. Additionally, the UI should be responsive, ensuring that it works well on all mobile phones, regardless the screen size.

The other user interface is the one provided to the CPOs, which are in charge of managing the charging service for the businesses involved. In this case the interface is a web application, which we also thought as easy to use, with a clear visualization in order for the CPOs to be able to keep track of all the charging stations and manage them properly. The UI in this case offers more complex features, also, allowing to the user to modify the graphical parts and personalize the aspects of the application. Exactly like for the mobile application, in this case we want a web application that allows a fast interaction without performance issues, and that works on any browser. 
