In this section, we briefly represent a list of the most important requirements of the \verb|eMAll|, remainig on a high level of abstraction, since we will proceed to further discuss about them in much more detail in the next chapter.
\subsection{Data collection and management}
One of the main functionalities of the software is to store and manage different kinds of data coming from different sources:
\begin{enumerate}
    \item The EVD using the eMma inserts into the system different kind of data. He inserts personal data, such as name, surname, and payment details; he also adds information about his EVs, like the maximum and minimum current supported, the connector type, the battery capacity and other relevant facts, like any additional EVSE he might own. The eMall allows the insertion of structured data and full-text elements that are subjected to checks in order to verify their correctness. The software maintains these data on the database in order to associate the bookings and the charging sessions to all registered EVDs, who can access all the functionalities of the system and are not subjected to the payment of a deposit every time they use a charging point
    \item The DSO provides energy to the charging stations, and the information about the DSO's supply is automatically collected by the CPMS subsystem of the eMall through interfaces that interact with the external systems. The CPMS acquires the information and saves it on the database in order for it to be visible to the CPO, and updates these data periodically. The collected data deriving from the DSO's are essential for the businesses, which make their supply choices depending on the price, the availability and the kind of acquired energy 
    \item The CPO manages the charging stations and their supply, visualizing the information kept by the software and making data-driven decisions for each one of the charging stations owned by the company. The CPO can see the parameters of each station and change them based on the new prices and types of energy, based on the chosen DSO to acquire from and based on the new politics of the CPO's company. All the data updates done by the CPO are received by the CPMS and collected by the system, so the managerial part of the service constantly produces data, about the charging stations. These data are stored and then used by the software to inform the EVDs of the charging stations details. The eMall also keeps data about the charging points and about the presence of batteries in each charging station, and these are useful information that need to be collected in order to allow to the CPOs to manage the service accurately
    \item The charging station itself is an important source of data. Information about the charging points usage, both in terms of frequency throughout different periods of time (day, week, month) and usage time (for how long a certain socket has been used for each charging process) must be kept to enable the eMall system to conduct data analysis procedures (can give information about peak load hours) and empower the CPOs with relevant data for the business decision making process. Other information that can be tracked through the system include: client profiling (keep track of clients who visit the charging station), maintenance record, unused bookings profile
\end{enumerate}

\subsection{Communication and knowledge sharing}
The eMall provides different tools to the EVD and to the CPO in order to take advantage of the service and obtain all the needed information from the system.. To be able to share this knowledge the subsystems of the software need to communicate among themselves and with the external entities. The offered tools are the following:
\begin{enumerate}
    \item The eMma presents to the EVD all the information needed about the nearby charging stations. The application shows a map with the charging stations, and selecting a station the user can visualize further data, such as price, socket type, free charging points and other details. The eMma and the eMci are able to provide these information, because are part of the eMSP, which communicates with the CPMS to acquire the data about the charging stations 
    \item The web app available to the CPO, communicates with the CPMS part of the software getting the data about the electric supply offered by the DSOs, acquiring knowledge about the prices, the special offers and the available electric sources. The CPMS updates the information interacting periodically with the external service of the DSOs and shares the knowledge with the CPO
\end{enumerate}
It is evident that among the functionalities of knowledge sharing and communication between the components involved, we also have as main features the following:
\begin{enumerate}
    \item The eMma shows to the EVD the information about the nearby charging stations
    \item The eMci shows to the EVD the data regarding the charging point in use
    \item The CPMS gives to the CPO the knowledge of the DSOs changes and the last data saved for each charging station managed by the CPO
\end{enumerate}

\subsection{Main functionalities}
Regarding the main functionalities that the EVD perceives, except for the ones already described, the most important ones remain:
\begin{enumerate}
    \item The eMma allows to the EVD to book a charging point in a chosen time frame. Once the booking is completed from the app, the EVD receives a confirmation notification and the booking with an associated code is added to the user history of charges. The system saves the data related to the registered EVD and to the booking, so the eMSP maintains a copy of the code provided to the user, the data associated to the charging station and the chosen time frame. The effective charging service will be provided when the user will correctly insert the received code into the che eMci of the specific charging point. The eMall, after checking the code, activates a charging session with respect to the EVD, having in this way that the system provides the functionality of charging the EV in the time frame previously booked
    \item The eMall gives, also, the possibility to charge without booking. In this case the EVD interacts  with the eMma and the eMci. From the two interfaces the data arrive to the eMSP, which creates the charging session and allows the user to use the service
\end{enumerate}