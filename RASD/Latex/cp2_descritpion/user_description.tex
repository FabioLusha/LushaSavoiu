The eMall has three main user classes:
\begin{enumerate}
    \item \textbf{Unregistered EVD}: An EVD can register to the eMall or use the service without registration. In order to register, the user has to introduce personal data and the details of the EVs, so he creates a profile with an associated name and a password. By creating a profile is possible to take advantage of all the features provided by the service, having some privileges, but the eMall can also be used without any registration. The eMma can be downloaded on the phone and used as a 'guest' and in this case is still possible to visualize the map with all the nearby stations and their information. It is also possible to book a charging session from the application, but is necessary at least the insertion of the payment details and the payment of a deposit in advance in order to use this functionality. Even in the case of charging the EV without any booking, the unregistered EVD has to give a deposit before starting the charging session. Furthermore, the EVD without a profile doesn't have the history of charges, so there are some limitations in using the system
    \item \textbf{Registered EVD}: An EVD is registered if creates an account inserting personal data and EVs details. The registered EVD interacts with the eMma and the eMci in order to use the main functionalities of the system: to book a charging session, to charge the EV without a booking, to visualize the nearby charging stations and to visualize and modify the personal profile and history. The EVD, registered or unregistered, can be unfamiliar with the use of mobile applications, so the software needs to be user-friendly in order to guarantee a good service in all its aspects 
    \item \textbf{CPO}: A company that supplies the service is identified with the employees or the existing software, that interacts with the eMall system. In the interaction the part of the company is called the CPO and manages the charging stations provided by the company itself. The CPO is able to visualize all the stations and the respective charging points and can change the supply parameters, modifying the price of the charge, the storage of energy, the DSOs from which to acquire electricity and other details. All these changes are possible given the interaction of the CPO with the CPMS part of the eMall, which has the necessary knowledge, that is communicated to the company in order to administer the stations and offer the service properly 
\end{enumerate}