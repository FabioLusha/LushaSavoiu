\subsection{Domain model}
We start off this chapter by analyzing the domain model (or \textit{conceptual model}) we came up with to represent the domain we are working in. Being a conceptual model the diagram was not drafted with all the formalism specified in the UML notation; we actually used a pretty informal description, specifying multiplicity and reading direction only when strictly necessary. The model in figure \ref{domain_model} was drawn using the UML class diagram notation, and illustrates only the conceptual classes that are significant to the domain \cite{larmanUML}. We also included the classes representing the component of our system to be that will interact directly with the environment.\\
\begin{figure}[H]
    \centering
    \includegraphics[width=1\textwidth]{Images/cp2/domain_model_v1}
    \caption{Domain model}
    \label{domain_model}
\end{figure}

We now proceed discussing some of the elements that may not be so immediate only by reading the domain model.


\paragraph{EVSE}
The EVSE, as mentioned in \ref{sec:Definitions, Acronyms, Abbreviations}, is a general term that refers to the the equipment that allows a vehicle to be recharged. In this context we can see it as a universal adapter that can be an interface for the vehicle inlet and the socket of the charging point. In the diagram this is shown by the association \textit{'Connects-to'} for both class EV and Socket. The multiplicity one-to-one for both associations is to show that at a specific moment a vehicle can be connected to a socket only through one single EVSE \cite{larmanUML}. Obviously, an EV can connect to all the EVSEs that have a connector compatible with its inlet, and a Socket can connect to all the EVSEs that have a compatible plug. It also should be noticed that there are charging points that do not expose the socket directly, but they have an integrated EVSE, namely a cable with it's connector.

\paragraph{DSO}
In the model the DSO is represented as interacting with both CPO and the CPMS. This happens because we do not distinguish between the actual business and the Information System he uses. So the association between DSO and CPMS is to be intended as the interaction between the two software systems, meanwhile, the association between DSO and CPO represents the interaction of the two businesses.

\paragraph{ChargingPoint}
This class represents the actual physical device that contains the sockets to which the EV will connect to charge.

\paragraph{Battery} In the association \textit{'Has'} between ChargingStation and Battery we have stated that an instance of ChargingStation may have only one Battery. The reason behind this decision is that we opted for a high level of abstraction and with the class Battery we intend a generic energy storage, without concerning ourselves with the actual physical properties or requirements of the item itself.

\paragraph{CPO} The multiplicity one-to-many in the association \textit{'Uses'} between CPO and CPMS is motivated by our view of the system. In our perspective, the CPMS is viewed as a software that is offered to different businesses that manage their charging stations, thus a business through the CPO can choose to manage the charging stations with different CPMS systems.

\subsection{State charts}
Among the main interactions with the eMall we have decided to represent here with state diagrams the ones that we consider the most interesting and complex uses of the system, from EVD point of view: the initiation of a charging session and the booking operation.

\begin{figure}[H]
    \centering
    \includegraphics[width=0.95\textwidth]{Images/ChargingNow.png}
    \caption{State diagram of the EVD that starts a charging session}
\end{figure}

\begin{figure}[H]
    \centering
    \includegraphics[width=0.85\textwidth]{Images/booking_charge_state_diag}
    \caption{State diagram of the EVD that books a charging point}
\end{figure}

We also report another state diagram to represent the main interaction of the CPO with the managerial part of our software. We consider the case in which the CPO wants to modify some parameters regarding a certain charging station. 

\begin{figure}[H]
    \centering
    \includegraphics[width=0.95\textwidth]{Images/CPOparams.png}
    \caption{State diagram of the CPO that manages charging stations}
\end{figure}

\subsection{Scenarios}
\paragraph{Booking a charging point}
Edward, after getting in his electric vehicle, notices that the battery is at low percentage, so he plans to book a charge at a station nearby. He grabs his smartphone and opens the mobile app eMma to look for a charging station. When Edward opens the app he is greeted with the view of a map showing him the nearest charging stations to his location. The charging stations are represented with icons of different colors. The colors are used to distinguish totally occupied charging stations from those with free sockets that can be used to charge an EV. After moving around the map, Edward finds a charging station with free sockets suitable to his needs. He clicks on the icon symbolizing the charging station and a new view is drawn on the screen. The view visualizes information about the specific station, namely:
\begin{itemize}
    \item the name of the charging station
    \item the rating of the charging station
    \item indication about the available sockets types
    \item contact details
    \item address of the charging station
    \item any directions on how to handle the charging process
    \item reviews relative to the charging station
\end{itemize}
Edward, being satisfied with the characteristics presented by the station, clicks on the button 'Book now' opening yet another view, which lists all the available sockets, pointing out the following information for each one of them:
\begin{itemize}
    \item the type of charging (AC/DC)
    \item the type of the socket (type 1, type 2, CCS, CHAdeMO, etc.)
    \item the charging speed denoted in kW and km/h (km gained per one hour of charge)
    \item the price for kWh
    \item the price for unlocking the socket
\end{itemize}
Scrolling down on the app, additional information regarding the station and the charging process are shown, like:
\begin{itemize}
    \item accessibility to the station
    \item any additional fees set by the CPO, such as the cost per minute for parking during the charging process and the cost per minute for parking after finishing the charging (penalty for occupying the spot and not using it)
    \item taxation information (VAT etc)
\end{itemize}
Being satisfied by the features offered by this station, Edward selects the appropriate socket for his EV and equipment (any additional charging adapter) and inserts the time-frame (date and time) in which he wants to reserve the spot.


\paragraph{Update profile details}
Jay is an electric vehicle enthusiast, who bought himself a new EV in order to reduce the negative impact on the environment. Given this new purchase, Jay needs to update his profile on the eMma in order to take advantage of the eMall service at its most. He logs in to his account on the mobile app and from the main page navigates to his profile. On the profile page are visible the personal information and the details about the EVs. On the page there is also the button that allows to update the profile and this is exactly what Jay is looking for. After pressing the button 'Update' there are different possibilities and Jay chooses the one that states 'Add new vehicle'. Now he has to fill up a form with the EV's details, such as type, capacity of the battery, supported power and electircal current and so on. After double checking the form, Jay presses the 'Ok' button and the page reloads showing again the profile page that now states among the other vehicles also the new one. Considering the eventuality of making mistakes in completing the form it is always possible to come back to the EV details and change any present field.

\paragraph{Visualize charging history}
It has been nearly 6 months since Hannah bought her first EV and now she has fully grasped how the whole ecosystem around it works. In the past few months Hannah has tried quite a few different charging stations to explore how each one is managed and organized. Having tried all these charging stations Hannah is curious to see how many charging stations she has visited, how frequently and how much she has paid for the charging. With this objective in mind Hannah opens the eMma application in her mobile phone and heads to the history section of it. This section is divided in two parts: in the top half the app shows the imminent charging booking that Hannah has reserved, if present. In the bottom half of the screen, the app shows a chronologically ordered list of all the charges processed through eMma. Each entry in the list shows the date in which the charging was done, where it was done, for how long, how many kWh were charged, the type of the socket used and how much did it cost distinguishing between cost for kWh and total cost. Hannah can also choose a different visualization for the charging history, filtering it based on some criteria (EV, time-period, location).

\paragraph{Start a charging session}
Adeline usually goes to the supermarket nearer to her house because it has a charging station in the parking area. Most of the times she finds an available charging point so she charges her EV while she does the grocery. Once stopped the car in the available spot Adeline wants to start a charging session. She interacts with the charging point interface, visualizes the information about the available charge with the respective power and cost and inserts the code shown on the screen in the mobile app of the eMall, the eMma. Once initiated the session from the mobile phone, the data are sent to the system, and in particular to the CPMS part of the software that checks the correctness of the inserted code and registers the session related to the user. Then, the CPMS enables the flow of electricity in order to actualize the charging from the charging point. If during the check of the data and during the activation operations there are no errors the status of the charging point changes and the session is activated. Now, Adeline can insert the specific plug, compatible with her car, in the EV in order to actually start the charging, which if not stopped earlier will terminate when the battery is full.

\paragraph{A new user registers to the system}
Michael, proud owner of an EV for 5 years, has decided to try this new charging app, eMma, that is promoting itself as a better alternative to manage in a smart way the charging process of an EV. Michael decides to give eMma a chance, downloads the app and immediately initiates the procedure to create a new account. The first phase of the registering process is straightforward; the usual information about name, surname, email and password are requested. After completing this first phase eMma prompts the user with a message asking him the consent to use his geographical location and to accept the terms of service. Micheal gladly accepts because he wants the app to show him the charging station nearest to him based on his location. After granting the consent a new page is presented to Michael, confirming the account creation. Now Micheal can access the system as a registered user but can use the functionalities to their fullest. To do so he must complete his profile by adding the details about his EVs and his payment details. To add details about his EVs Micheal has to complete a form where he can enter the EVs specifications and his EV equipment, like charging adapters and cables. Michael understands that this information is needed so the app can work in a smart way, showing him only the charging stations that have sockets compatible with Michael EVs connector or adapters.

\paragraph{Visualize the charging stations map}
Daisy is an unusual user of the eMall, that didn't registered an account. Anyway the system allows the possibility to use the application as a guest, but the functionalities are limited. Daisy is only interested in visualizing the charging stations nearby, so she opens the main page of the app in order to look at the map. The system retrieves, based on the location shared by the phone, the charging stations in the area and shows them on the map. Daisy can now explore the charging stations around clicking on them on the map, and she can see their rating with the relative reviews and can choose the service that better fits her needs. She can visualize the price and the available chargers with their type of connectors, but she is not able to book a charging session without an account. Once identified her preferred charging station Daisy closes the application, gets in her car and heads directly to charge her EV.

\paragraph{Manage the charging stations}
Nick is a CPO that on a typical day has to monitor the charging stations assigned to his department. After arriving to work and logging in the system with the company credentials Nick checks the list of charging stations and any new notification given by the CPMS part of the eMall about the DSOs decisions. He sees a warning regarding the recently deployed charging station in Rome and clicking on it the web application of the eMall shows a form with the various characteristics of the station. The parts that may have undergone a modification are highlighted in red and in this case the selection of the DSO has new options available and Nick clicks on it to explore the more convenient ones. He notices a change in one of the DSOs that now grants energy also through renewable resources, and given the green policy of the company he chooses this new kind of supply. Considering the price of 0.036/kWh provided by the DSO, in order to have a profit, according to the business modus operandi Nick sets the cost charging at 0.040/kWh. Once confirmed the DSO from which to acquire energy by looking carefully at the form he becomes aware of the fact that at the moment there are no EVD charging at the station and also sees that there are available batteries in which to store energy. One of the batteries is empty, so he selects it in order to store energy in it right away, until the full capacity C, given the off-peak moment. After all this operations Nick saves the changes and the eMall notifies him about the success of the procedure, that has an immediate effect on the system and his interaction with the world. Nick moves on to the next charging station of his list, checking up in similar way each one of the stations for which he is responsible. 

